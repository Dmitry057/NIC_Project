\documentclass[conference]{IEEEtran}
\IEEEoverridecommandlockouts
% The preceding line is only needed to identify funding in the first footnote. If that is unneeded, please comment it out.
%Template version as of 6/27/2024

\usepackage{cite}
\usepackage{amsmath,amssymb,amsfonts}
\usepackage{algorithmic}
\usepackage{graphicx}
\usepackage{textcomp}
\usepackage{xcolor}
\usepackage{hyperref} % Added to support clickable links
\def\BibTeX{{\rm B\kern-.05em{\sc i\kern-.025em b}\kern-.08em
    T\kern-.1667em\lower.7ex\hbox{E}\kern-.125emX}}
\begin{document}

\title{Nature Inspired Computing Research Proposal:\\ Transaction Fraud Detection \\ \vspace{0.5em} Group Members: Nikita Zagainov, Dmitry Tetkin, Alisher Kamolov, Nikita Tsukanov}

\author{
    \IEEEauthorblockN{Nikita Zagainov}
    \IEEEauthorblockA{
        \textit{Innopolis University}\\
        Innopolis\\
        n.zagainov@innopolis.university
    }
    \and
    \IEEEauthorblockN{Dmitry Tetkin}
    \IEEEauthorblockA{
        \textit{Innopolis University}\\
        Innopolis\\
        d.tetkin@innopolis.university
    }
    \and
    \IEEEauthorblockN{Alisher Kamolov}
    \IEEEauthorblockA{
        \textit{Innopolis University}\\
        Innopolis\\
        a.kamolov@innopolis.university
    }
    \and
    \IEEEauthorblockN{Nikita Tsukanov}
    \IEEEauthorblockA{
        \textit{Innopolis University}\\
        Innopolis\\
        n.thukanov@innopolis.university
    }
}

\maketitle

\begin{abstract}
This proposal outlines our approach to Transaction Fraud Detection using a combination of nature inspired computing and machine learning techniques. Our goal is to develop a robust system that efficiently identifies fraudulent transactions.
\end{abstract}

\begin{IEEEkeywords}
Nature Inspired Computing, Fraud Detection, Machine Learning.
\end{IEEEkeywords}

\section{Project Proposal}

\subsection{Project Idea}
Our project focuses on detecting fraudulent transactions in financial systems. The system combines nature inspired computing algorithms with classical machine learning to improve detection accuracy.

\subsection{Method/Technique}
We propose a hybrid approach that employs natural phenomena-inspired algorithms (e.g., genetic algorithms, swarm intelligence) alongside supervised learning models to optimize both feature selection and model performance.

\subsection{Dataset}
The primary dataset is available on Kaggle: \href{https://www.kaggle.com/c/ieee-fraud-detecon/overview}{https://www.kaggle.com/c/ieee-fraud-detecon/overview}. Additional implementation details and resources are provided in the related GitHub repository: \href{https://github.com/pmacinec/transacons-fraud-detecon}{https://github.com/pmacinec/transacons-fraud-detecon}.

\subsection{Timeline}
\begin{itemize}
    \item \textbf{Week 1-2:} 
    \begin{itemize}
        \item Project kickoff and requirement gathering.
        \item Initial dataset inspection and preprocessing setup.
    \end{itemize}
    \item \textbf{Week 3-4:} 
    \begin{itemize}
        \item Development and prototyping of nature inspired computing algorithms.
        \item Setup of baseline machine learning models.
    \end{itemize}
    \item \textbf{Week 5-6:} 
    \begin{itemize}
        \item Integration of the hybrid approach.
        \item Continuous testing and refinement of algorithms and models.
    \end{itemize}
    \item \textbf{Week 7-8:} 
    \begin{itemize}
        \item Comprehensive system testing and performance evaluation.
        \item Documentation, final report preparation, and project presentation.
    \end{itemize}
\end{itemize}

\subsection{Individual Contributions}
\begin{itemize}
    \item \textbf{Nikita Zagainov:}
    \begin{itemize}
        \item Data preprocessing and exploratory data analysis.
        \item Building the initial data cleaning and transformation pipelines.
    \end{itemize}
    \item \textbf{Dmitry Tetkin:}
    \begin{itemize}
        \item Design and prototyping of nature inspired computing algorithms.
        \item Iterative improvement and integration of algorithms throughout the timeline.
    \end{itemize}
    \item \textbf{Alisher Kamolov:}
    \begin{itemize}
        \item Implementation and tuning of machine learning models.
        \item Integration of model outputs with nature inspired methods.
    \end{itemize}
    \item \textbf{Nikita Tsukanov:}
    \begin{itemize}
        \item End-to-end system testing and performance evaluation.
        \item Comprehensive documentation and final report preparation.
    \end{itemize}
\end{itemize}

\subsection{References}
\begin{thebibliography}{00}
\bibitem{dataset} Kaggle, ``IEEE Fraud Detection,'' Available: \href{https://www.kaggle.com/c/ieee-fraud-detecon/overview}{https://www.kaggle.com/c/ieee-fraud-detecon/overview}.
\bibitem{github} GitHub, ``Transacons Fraud Detection,'' Available: \href{https://github.com/pmacinec/transacons-fraud-detecon}{https://github.com/pmacinec/transacons-fraud-detecon}.
\end{thebibliography}

\vspace{12pt}

\end{document}
